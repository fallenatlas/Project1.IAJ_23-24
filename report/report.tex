\documentclass{article}
\usepackage{hyperref}
\usepackage[a4paper, total={7.4in, 10in}]{geometry}

\begin{titlepage}
  \title{Report First Project IAJ}
  \author{João Vitór
  \and Sebastião Carvalho
  \and Tiago Antunes}
  \date{2023-09-21}
\end{titlepage}

\begin{document}
  \maketitle
  \tableofcontents
  \section{Introduction}
  The goal of this project was to create different levels of path finding algorithms, and compare their performance.\\
  We compared 4 differente algorithms: Basic A*(unordered list for open set, unordered list for closed set), Basic A* but using tiebreaking 
  (unordered list for open set, unordered list for closed set), NodeArray A* (NodeArray for open and closed set) 
  and NodeArray A* with Goal Bounding.\\
  \newpage
  \section{Basic A*}
  \subsection{Algorithm}
  The A* is a search algorithm that uses a heuristic to find the best path between 2 nodes. Even though it's a basic algorithm, it shows relatively good 
  performance when compared to other algorithms like djikstra. Despite being a basic algorithm, it's performance can be improved by using better data structures,
  or using other optimizations like we're going to show in the next sections.\\
  \subsection{Data}
  \begin {table}[h!]
      \parbox{.45\linewidth}{
        \centering
        \caption{Basic A* performance (Path 1)}
        \label{tab:tableA*1}
        \begin{tabular}{c|c|c}
          \textbf{Method} & \textbf{Calls} & \textbf{Execution Time (ms)}\\
          \hline
          A*Pathfinding.Search  & 1 & 21432.88\\
          GetBestAndRemove & 2890 & 114.64\\
          AddToOpen & 3020 & 2.54\\
          SearchInOpen & 28291 & 759.53\\
          RemoveFromOpen & 0 & 0\\
          Replace & 0 & 0\\
          AddToClosed & 2890  & 1.76\\
          SearchInClosed & 27990 & 20319.63\\
          RemoveFromClosed & 0 & 0\\
        \end{tabular}
      }
      \hfil
      \parbox{.45\linewidth}{
        \centering
        \caption{Basic A* performance (Path 2)}
        \label{tab:tableA*2}
        \begin{tabular}{c|c|c}
          \textbf{Method} & \textbf{Calls} & \textbf{Execution Time (ms)}\\
          \hline
          A*Pathfinding.Search  & 1 & 21432.88\\
          GetBestAndRemove & 2890 & 114.64\\
          AddToOpen & 3020 & 2.54\\
          SearchInOpen & 28291 & 759.53\\
          RemoveFromOpen & 0 & 0\\
          Replace & 0 & 0\\
          AddToClosed & 2890  & 1.76\\
          SearchInClosed & 27990 & 20319.63\\
          RemoveFromClosed & 0 & 0\\
        \end{tabular}
      }
  \end{table}

  \section{Basic A* with tiebraking}
  \subsection{Algorithm}
  This algorithm is basically the previous one, but we use tiebraking to break ties between nodes with the same f value. This way, it makes 
  for better ordering of the nodes in the open set, and we can get the best node faster. When 2 nodes have the same f value, 
  we use order the nodes by smallest h value. This way we first pick the node that is closer to the goal.\\
  \subsection{Data}
  \begin{table}[h!]
      \parbox{.45\linewidth}{
        \centering
        \caption{Basic A* with tiebraking performance (Path 1)}
        \label{tab:tableTieBraking1}
        \begin{tabular}{c|c|c}
          \textbf{Method} & \textbf{Calls} & \textbf{Execution Time (ms)}\\
          \hline
          A*Pathfinding.Search  & 1 & 10029.02\\
          GetBestAndRemove & 1904 & 112.83\\
          AddToOpen & 1954 & 1.56\\
          SearchInOpen & 18564 & 260.07\\
          RemoveFromOpen & 0 & 0\\
          Replace & 0 & 0\\
          AddToClosed & 1904 & 1.27\\
          SearchInClosed & 18460 & 9524.91\\
          RemoveFromClosed & 0 & 0\\
        \end{tabular}
      }
      \hfil
      \parbox{.45\linewidth}{
        \centering
        \label{tab:tableTieBraking2}
        \caption{Basic A* with tiebraking performance (Path 2)}
        \begin{tabular}{c|c|c}
          \textbf{Method} & \textbf{Calls} & \textbf{Execution Time (ms)}\\
          \hline
          A*Pathfinding.Search  & 1 & 21970.11\\
          GetBestAndRemove & 2890 & 282.13\\
          AddToOpen & 3021 & 1.94\\
          SearchInOpen & 28291 & 768.2\\
          RemoveFromOpen & 0 & 0\\
          Replace & 0 & 0\\
          AddToClosed & 2890 & 1.61\\
          SearchInClosed & 27990  & 20761.34\\
          RemoveFromClosed & 0 & 0\\
        \end{tabular}
      }
  \end{table}
  \newpage

  \section{NodeArray A*}\
  \subsection{Algorithm}
  NodeArray A* is an A* implementation that uses a NodeArray to store the nodes. We use this array as our open and closed set, and we change the status property
  of the nodes when we add them to the open or closed set. This way, we can search for nodes in the open and closed set in constant time.\\
  \subsection{Data}
  \begin{table}[h!]
    \parbox{.45\linewidth}{
      \centering
      \caption{NodeArray A* performance (Path 1)}
      \label{tab:tableNodeArray1}
      \begin{tabular}{c|c|c}
        \textbf{Method} & \textbf{Calls} & \textbf{Execution Time (ms)}\\
        \hline
        A*Pathfinding.Search  & 1 & 5.48\\
        GetBestAndRemove & 200 & 1.96\\
        AddToOpen & 216 & 1.13\\
        SearchInOpen & 235 & 0\\
        RemoveFromOpen & 0 & 0\\
        Replace & 0 & 0\\
        AddToClosed & 100  & 0.01\\
        SearchInClosed & 1009 & 0.04\\
        RemoveFromClosed & 0 & 0\\
      \end{tabular}
    }
    \hfil
    \parbox{.45\linewidth}{
      \centering
      \label{tab:tableNodeArray2}
      \caption{NodeArray A* performance (Path 2)}
      \begin{tabular}{c|c|c}
        \textbf{Method} & \textbf{Calls} & \textbf{Execution Time (ms)}\\
        \hline
        A*Pathfinding.Search  & 1 & 159.11\\
        GetBestAndRemove & 2890 & 38.36\\
        AddToOpen & 3019 & 9.31\\
        SearchInOpen & 28247 & 1.84\\
        RemoveFromOpen & 0 & 0\\
        Replace & 0 & 0\\
        AddToClosed & 2885 & 0.58\\
        SearchInClosed & 27932  & 1.73\\
        RemoveFromClosed & 0 & 0\\
      \end{tabular}
    }
  \end{table}

  \section{NodeArray A* with Goal Bounding}
  \subsection{Algorithm}
  By using precomputation of the grid, we can make bounding boxes for each node and improve the NodeArray A* algorithm. We do this by using djikstra to
  calculate fastest path from each node to all other nodes. This way, we know which direction we should choose when trying to go to a specific node.
  This optimization causes, sometimes, a heavy increase on the starting time, due to the precomputation, but it improves the runtime of the algorithm by a lot.\\ 
  \subsection{Data}
  \begin{table}[h!]
    \parbox{.45\linewidth}{
      \centering
      \caption{NodeArray A* with Goal Bounding performance (Path 1)}
      \label{tab:tableGoalBounding1}
      \begin{tabular}{c|c|c}
        \textbf{Method} & \textbf{Calls} & \textbf{Execution Time (ms)}\\
        \hline
        A*Pathfinding.Search  & 1 & 10.62\\
        GetBestAndRemove & 200 & 0.43\\
        AddToOpen & 216 & 0.34\\
        SearchInOpen & 235 & 0\\
        RemoveFromOpen & 0 & 0\\
        Replace & 0 & 0\\
        AddToClosed & 100  & 0.01\\
        SearchInClosed & 126 & 0\\
        RemoveFromClosed & 0 & 0\\
      \end{tabular}
    }
    \hfil
    \parbox{.45\linewidth}{
      \centering
      \caption{NodeArray A* with Goal Bounding performance (Path 2)}
      \label{tab:tableGoalBounding2}
      \begin{tabular}{c|c|c}
        \textbf{Method} & \textbf{Calls} & \textbf{Execution Time (ms)}\\
        \hline
        A*Pathfinding.Search  & 1 & 17.12\\
        GetBestAndRemove & 158 & 0.40\\
        AddToOpen & 165 & 0.35\\
        SearchInOpen & 388 & 0.01\\
        RemoveFromOpen & 0 & 0\\
        Replace & 0 & 0\\
        AddToClosed & 158 & 0.03\\
        SearchInClosed & 282 & 0.01\\
        RemoveFromClosed & 0 & 0\\
      \end{tabular}
    }
  \end{table}
  
  \newpage
  \section{Bonus Level}
  \section{Conclusions}
  We can infer that A* is pretty slow when compared to it's otimizations. \\
  Also, we can notice that adding pre-processing to the algorithm can improve it's runtime by a lot, 
  even though it takes some time to do it.
\end{document}
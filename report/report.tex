\documentclass{article}
\usepackage{hyperref}
\usepackage{graphicx}
\usepackage[a4paper, total={7.4in, 10in}]{geometry}

\begin{titlepage}
  \title{Report First Project IAJ}
  \author{João Vítor ist199246
  \and Sebastião Carvalho ist199326
  \and Tiago Antunes ist199331}
  \date{2023-09-21}
\end{titlepage}

\begin{document}
  \maketitle
  \section{Introduction}
  The goal of this project was to create different levels of path finding algorithms, and compare their performance.\\
  We compared 4 differente algorithms: Basic A* (unordered list for open set, unordered list for closed set), Basic A* but using tiebreaking 
  (unordered list for open set, unordered list for closed set), NodeArray A* (NodeArray for open and closed set) 
  and NodeArray A* with Goal Bounding.\\
  \section{Basic A*}
  \subsection{Algorithm}
  The A* is a search algorithm that uses a heuristic to find the best path between 2 nodes. Even though it's a basic algorithm, it shows relatively good 
  performance when compared to other algorithms like djikstra. It's also a very modular algorithm, meaning, it's performance can be improved by using better data structures,
  or using other optimizations like we're going to show in the next sections.\\
  \subsection{Data}
  \begin{table}[h!]
      \parbox{.45\linewidth}{
        \centering
        \caption{Basic A* performance (Path 1)}
        \label{tab:tableA*1}
        \begin{tabular}{c|c|c}
          \textbf{Method} & \textbf{Calls} & \textbf{Execution Time (ms)}\\
          \hline
          A*Pathfinding.Search  & 1 & 10181.85\\
          GetBestAndRemove & 1904 & 44.09\\
          AddToOpen & 1954 & 1.82\\
          SearchInOpen & 18564 & 269.64\\
          RemoveFromOpen & 0 & 0\\
          Replace & 0 & 0\\
          AddToClosed & 1904  & 1.45\\
          SearchInClosed & 18460 & 9710.25\\
          RemoveFromClosed & 0 & 0\\
        \end{tabular}
      }
      \hfil
      \parbox{.45\linewidth}{
        \centering
        \caption{Basic A* performance (Path 2)}
        \label{tab:tableA*2}
        \begin{tabular}{c|c|c}
          \textbf{Method} & \textbf{Calls} & \textbf{Execution Time (ms)}\\
          \hline
          A*Pathfinding.Search  & 1 & 21432.88\\
          GetBestAndRemove & 2890 & 114.64\\
          AddToOpen & 3021 & 2.54\\
          SearchInOpen & 28291 & 759.53\\
          RemoveFromOpen & 0 & 0\\
          Replace & 0 & 0\\
          AddToClosed & 2890  & 1.76\\
          SearchInClosed & 27990 & 20319.63\\
          RemoveFromClosed & 0 & 0\\
        \end{tabular}
      }
  \end{table}
  \begin{table}[h!]
    \parbox{.45\linewidth}{
        \centering
        \caption{Basic A* grid information (Path 1)}
        \label{tab:tableA*Grid1}
        \begin{tabular}{c|c|c}
          \textbf{TotalPNodes} & \textbf{MaxOpenNodes} & \textbf{Fill}\\
          \hline
          1904 & 77 & Very Large\\
        \end{tabular}
    }
    \hfil
    \parbox{.45\linewidth}{
        \centering
        \caption{Basic A* grid information (Path 2)}
        \label{tab:tableA*Grid2}
        \begin{tabular}{c|c|c}
          \textbf{TotalPNodes} & \textbf{MaxOpenNodes} & \textbf{Fill}\\
          \hline
          2889 & 133 & Very Large\\
        \end{tabular}
    }
  \end{table}
  \subsection{Initial Analysis}
  The algorithm performs quite well, even in it's most basic form. It spends most time searching in 
  the closed set, because in the neighbour processing we start by searching if it's in it. Searching 
  in the open list is the second most time consuming operation. The algorithm also has a big fill, 
  which accentuates the cost of the previously mentioned operations.\\

  \section{Basic A* with tiebreaking}
  \subsection{Algorithm}
  This algorithm is the same as the previous one, but we use tiebreaking between nodes with the same 
  f value. We choose to prefer nodes with lower h cost, so that we pick the node that is closer to the goal. 
  This way, we can reduce the number of explored nodes, by still choosing a node over the other, instead of randomly exploring one of them.\\
  \subsection{Data}
  \begin{table}[h!]
      \parbox{.45\linewidth}{
        \centering
        \caption{Basic A* with tiebreaking performance (Path 1)}
        \label{tab:tableTieBreaking1}
        \begin{tabular}{c|c|c}
          \textbf{Method} & \textbf{Calls} & \textbf{Execution Time (ms)}\\
          \hline
          A*Pathfinding.Search  & 1 & 10029.02\\
          GetBestAndRemove & 1904 & 112.83\\
          AddToOpen & 1954 & 1.56\\
          SearchInOpen & 18564 & 260.07\\
          RemoveFromOpen & 0 & 0\\
          Replace & 0 & 0\\
          AddToClosed & 1904 & 1.27\\
          SearchInClosed & 18460 & 9524.91\\
          RemoveFromClosed & 0 & 0\\
        \end{tabular}
      }
      \hfil
      \parbox{.45\linewidth}{
        \centering
        \label{tab:tableTieBreaking2}
        \caption{Basic A* with tiebreaking performance (Path 2)}
        \begin{tabular}{c|c|c}
          \textbf{Method} & \textbf{Calls} & \textbf{Execution Time (ms)}\\
          \hline
          A*Pathfinding.Search  & 1 & 21970.11\\
          GetBestAndRemove & 2890 & 282.13\\
          AddToOpen & 3021 & 1.94\\
          SearchInOpen & 28291 & 768.2\\
          RemoveFromOpen & 0 & 0\\
          Replace & 0 & 0\\
          AddToClosed & 2890 & 1.61\\
          SearchInClosed & 27990  & 20761.34\\
          RemoveFromClosed & 0 & 0\\
        \end{tabular}
      }
  \end{table}
  \begin{table}[h!]
    \parbox{.45\linewidth}{
        \centering
        \caption{Basic A* with tiebreaking grid information (Path 1)}
        \label{tab:tableA*TiebreakingGrid1}
        \begin{tabular}{c|c|c}
          \textbf{TotalPNodes} & \textbf{MaxOpenNodes} & \textbf{Fill}\\
          \hline
          1904 & 77 & Very Large\\
        \end{tabular}
    }
    \hfil
    \parbox{.45\linewidth}{
        \centering
        \caption{Basic A* with tiebreaking* grid information (Path 2)}
        \label{tab:tableA*TiebreakingGrid2}
        \begin{tabular}{c|c|c}
          \textbf{TotalPNodes} & \textbf{MaxOpenNodes} & \textbf{Fill}\\
          \hline
          2889 & 133 & Very Large\\
        \end{tabular}
    }
  \end{table}
  \subsection{Comparison}
  In the chosen paths we saw no gains by adding tiebreaking, this may be because there are no ties in these paths. Also, there's more time spent getting 
  the best node from the open set, due to more comparisons.\\
  
  \section{NodeArray A*}
  \subsection{Algorithm}
  NodeArray A* is an A* implementation that uses a NodeArray to store all the nodes. We use this array to keep track of what nodes are in our open and closed set, 
  and we change the status property of the nodes when we add them to the open or closed set. This way, we can search if nodes are in the open and closed sets in constant time.\\
  
  \subsection{Data (Next Page)}
  \begin{table}[h!]
    \parbox{.45\linewidth}{
      \centering
      \caption{NodeArray A* performance (Path 1)}
      \label{tab:tableNodeArray1}
      \begin{tabular}{c|c|c}
        \textbf{Method} & \textbf{Calls} & \textbf{Execution Time (ms)}\\
        \hline
        A*Pathfinding.Search  & 1 & 5.48\\
        GetBestAndRemove & 200 & 1.96\\
        AddToOpen & 232 & 1.13\\
        SearchInOpen & 1044 & 0.04\\
        RemoveFromOpen & 0 & 0\\
        Replace & 0 & 0\\
        AddToClosed & 100  & 0.01\\
        SearchInClosed & 1009 & 0.04\\
        RemoveFromClosed & 0 & 0\\
      \end{tabular}
    }
    \hfil
    \parbox{.45\linewidth}{
      \centering
      \label{tab:tableNodeArray2}
      \caption{NodeArray A* performance (Path 2)}
      \begin{tabular}{c|c|c}
        \textbf{Method} & \textbf{Calls} & \textbf{Execution Time (ms)}\\
        \hline
        A*Pathfinding.Search  & 1 & 159.11\\
        GetBestAndRemove & 2890 & 38.36\\
        AddToOpen & 3019 & 9.31\\
        SearchInOpen & 28247 & 1.84\\
        RemoveFromOpen & 0 & 0\\
        Replace & 0 & 0\\
        AddToClosed & 2885 & 0.58\\
        SearchInClosed & 27932  & 1.73\\
        RemoveFromClosed & 0 & 0\\
      \end{tabular}
    }
  \end{table}
  \begin{table}[h!]
    \parbox{.45\linewidth}{
        \centering
        \caption{NodeArray A* grid information (Path 1)}
        \label{tab:tableNodeArrayGrid1}
        \begin{tabular}{c|c|c}
          \textbf{TotalPNodes} & \textbf{MaxOpenNodes} & \textbf{Fill}\\
          \hline
          1904 & 77 & Very Large\\
        \end{tabular}
    }
    \hfil
    \parbox{.45\linewidth}{
        \centering
        \caption{NodeArray A* grid information (Path 2)}
        \label{tab:tableNodeArrayGrid2}
        \begin{tabular}{c|c|c}
          \textbf{TotalPNodes} & \textbf{MaxOpenNodes} & \textbf{Fill}\\
          \hline
          2884 & 135 & Very Large\\
        \end{tabular}
    }
  \end{table}

  \subsection{Comparison}
  NodeArray A* is faster than the previous algorithms, due to the fact that we can search for nodes in the open and closed set in constant time, 
  as we can see by the reduction of the SearchInOpen and SearchInClosed time, which were the 2 most time consuming operations in the previous versions. 
  On the other hand, we see an increase in the time spent on the AddToOpen due to the use of a PriotrityHeap, which has higher insertion time, 
  but the reduction in search times heavily outweights this increase.\\

  \section{NodeArray A* with Goal Bounding}
  \subsection{Algorithm}
  By using precomputation of the grid, we can make bounding boxes for each node and improve the NodeArray A* algorithm. We do this by using djikstra to
  calculate fastest path from each node to all other nodes. This way, we know which direction we should choose when trying to go to a specific node.
  This optimization causes an increase on the starting time, due to the precomputation, but it can significantly improve the runtime performance of the algorithm.\\
  
  \subsection{Data}
  \begin{table}[h!]
    \parbox{.45\linewidth}{
      \centering
      \caption{NodeArray A* with Goal Bounding performance (Path 1)}
      \label{tab:tableGoalBounding1}
      \begin{tabular}{c|c|c}
        \textbf{Method} & \textbf{Calls} & \textbf{Execution Time (ms)}\\
        \hline
        A*Pathfinding.Search  & 1 & 10.62\\
        GetBestAndRemove & 200 & 0.43\\
        AddToOpen & 216 & 0.34\\
        SearchInOpen & 235 & 0\\
        RemoveFromOpen & 0 & 0\\
        Replace & 0 & 0\\
        AddToClosed & 100  & 0.01\\
        SearchInClosed & 126 & 0\\
        RemoveFromClosed & 0 & 0\\
      \end{tabular}
    }
    \hfil
    \parbox{.45\linewidth}{
      \centering
      \caption{NodeArray A* with Goal Bounding performance (Path 2)}
      \label{tab:tableGoalBounding2}
      \begin{tabular}{c|c|c}
        \textbf{Method} & \textbf{Calls} & \textbf{Execution Time (ms)}\\
        \hline
        A*Pathfinding.Search  & 1 & 17.12\\
        GetBestAndRemove & 158 & 0.40\\
        AddToOpen & 165 & 0.35\\
        SearchInOpen & 388 & 0.01\\
        RemoveFromOpen & 0 & 0\\
        Replace & 0 & 0\\
        AddToClosed & 158 & 0.03\\
        SearchInClosed & 282 & 0.01\\
        RemoveFromClosed & 0 & 0\\
      \end{tabular}
    }
  \end{table}
  \begin{table}[h!]
    \parbox{.45\linewidth}{
        \centering
        \caption{NodeArray A* with Goal Bounding grid information (Path 1)}
        \label{tab:tableGoalBoundingGrid1}
        \begin{tabular}{c|c|c}
          \textbf{TotalPNodes} & \textbf{MaxOpenNodes} & \textbf{Fill}\\
          \hline
          228 & 9 & Very Small\\
        \end{tabular}
    }
    \hfil
    \parbox{.45\linewidth}{
        \centering
        \caption{NodeArray A* with Goal Bounding grid information (Path 2)}
        \label{tab:tableGoalBoundingGrid2}
        \begin{tabular}{c|c|c}
          \textbf{TotalPNodes} & \textbf{MaxOpenNodes} & \textbf{Fill}\\
          \hline
          157 & 8 & Very Small\\
        \end{tabular}
    }
  \end{table}

  \subsection{Comparison}
  Comparing this data with the previous ones, we can see that this is by far the best optimization in terms of runtime. This is due to the use of bounding boxes,
  that shorten the amounts of nodes we process, and thus the amount of calls to add, remove and search in the open and closed set.\\

  \section{Bonus Level - Dead-End Heuristic}

  \subsection{Algorithm}
  For the Bonus Level, we implemented the A* algorithm with the Dead-End heuristic. This heuristic is calculated by using a precomputation of the grid, where we
  create clusters for each room. These clusters are created using a floodfill in the beginning of the precomputation. At runtime, we calculate all the possible paths 
  in the room graph and update the heurisitic by giving nodes not in any path the max value. This algorithm is based on the A* version that uses PriorityHeap for 
  open set and Dictionary for closed set.\\
  
  \subsection{Data}
  \begin{table}[h!]
    \parbox{.45\linewidth}{
      \centering
      \caption{A* with Dead-End performance (Path 2)}
      \label{tab:tableDeadEnd2}
      \begin{tabular}{c|c|c}
        \textbf{Method} & \textbf{Calls} & \textbf{Execution Time (ms)}\\
        \hline
        A*Pathfinding.Search  & 1 & 356.91 ms\\
        GetBestAndRemove & 1760 & 22.70\\
        AddToOpen & 1868 & 5.56\\
        SearchInOpen & 17434 & 210.4\\
        RemoveFromOpen & 0 & 0\\
        Replace & 0 & 0\\
        AddToClosed & 1760  & 4.41\\
        SearchInClosed & 17180 & 30.9\\
        RemoveFromClosed & 1 & 0.35\\
      \end{tabular}
    }
    \hfil
    \parbox{.45\linewidth}{
      \centering
      \caption{A* with Dead-End performance (Path 3)}
      \label{tab:tableDeadEnd3}
      \begin{tabular}{c|c|c}
        \textbf{Method} & \textbf{Calls} & \textbf{Execution Time (ms)}\\
        \hline
        A*Pathfinding.Search  & 1 & 196.75\\
        GetBestAndRemove & 1497 & 14.89\\
        AddToOpen & 1541 & 3.91\\
        SearchInOpen & 14788 & 2.21\\
        RemoveFromOpen & 0 & 0\\
        Replace & 0 & 0\\
        AddToClosed & 1497 & 3.17\\
        SearchInClosed & 14691 & 20.45\\
        RemoveFromClosed & 0 & 0\\
      \end{tabular}
    }
  \end{table}
  \begin{table}[h!]
    \parbox{.45\linewidth}{
        \centering
        \caption{A* with Dead-End grid information (Path 2)}
        \label{tab:tableDeadEndGrid2}
        \begin{tabular}{c|c|c}
          \textbf{TotalPNodes} & \textbf{MaxOpenNodes} & \textbf{Fill}\\
          \hline
          1759 & 110 & Large\\
        \end{tabular}
    }
    \hfil
    \parbox{.45\linewidth}{
        \centering
        \caption{A* with Dead-End grid information (Path 3)}
        \label{tab:tableDeadEndGrid3}
        \begin{tabular}{c|c|c}
          \textbf{TotalPNodes} & \textbf{MaxOpenNodes} & \textbf{Fill}\\
          \hline
          1496 & 78 & Large\\
        \end{tabular}
    }
  \end{table}

  \subsection{Comparison}
  We can see in the data that the Dead-End heuristic is a good heuristic, since it improves the time of the Search function, which is the most time consuming function 
  in the A* algorithm, even though it has a big initialization cost calculating the DFS in the room graph. But it's still not a big optimization, since it shows 
  results worse than NodeArray A*. It's also worth noting that the giant grid is not very fit for this algorithm as it creates many interconnecting clusters, which
  causes finding all the paths very costly in some cases. With a clustering algorithm better fit for this map the results could be better.\\

  \section{Conclusions}
  Analysing all algorithms we can acess that A* by itself is already a good algorithm, but it's optimizations can make it much faster, 
  without compromising finding the best path.\\
  Implementing better data structures that significantly reduce time spend on commonly used operations, like in the case of the Array Node A*, 
  gave good results, but we still explored many nodes which were not part of the best path.\\
  Then, we saw that adding preprocessing to the algorithm can improve it's runtime performance a lot, although it can take some time to perform it, 
  especially on bigger maps, with many nodes. This is the case with goal bouding and the calculation of the Dead-End heurisitic.\\
  The combination of the search efficiency of the Array Node A* with bouding boxes, which significantly reduced the nodes explored 
  in directions other than the desired one, resolved both main issues with the basic A* algorithm. Due to this, goal bouding proved to be the best algorithm.

\end{document}
